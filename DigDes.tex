% !TeX program = xelatex
% !TeX encoding = utf8
% !TeX root = DigDes.tex

%% TODO: publish to CTAN
\documentclass[margin=small]{tex/hsrzf}

%%%%%%%%%%%%%%%%%%%%%%%%%%%%%%%%%%%%%%%%%%%%%%%%%%%
% Packages

%% TODO: publish to CTAN
\usepackage{tex/hsrstud}
\usepackage{tex/docmacros}

%% Language configuration
\usepackage{polyglossia}
\setdefaultlanguage{english}

%% Pretty drawings
\usepackage{tikz}

\usetikzlibrary{calc}
\usetikzlibrary{positioning}

\usetikzlibrary{external}
\tikzexternalize[
  mode = graphics if exists,
  figure list = true,
  prefix=build/
]

%% Tweak lists
\usepackage{enumitem}
\setitemize{noitemsep}

%% Nice tables
\usepackage{array}
\usepackage{tabularx}
\usepackage{booktabs}

%% License configuration
\usepackage[
    type={CC},
    modifier={by-nc-sa},
    version={4.0},
    % lang={german},
]{doclicense}

%%%%%%%%%%%%%%%%%%%%%%%%%%%%%%%%%%%%%%%%%%%%%%%%%%%
% Metadata

\course{Elektrotechnik}
\module{DigDes}
\semester{Spring Semester 2021}

\authoremail{naoki.pross@ost.ch}
\author{\textsl{Naoki Pross} -- \texttt{\theauthoremail}}

\title{\texttt{\themodule}: Digital Design}
\date{\thesemester}

\setlength{\droptitle}{-1.3cm}

%%%%%%%%%%%%%%%%%%%%%%%%%%%%%%%%%%%%%%%%%%%%%%%%%%%
% Document

\begin{document}

\pagenumbering{roman}
\maketitle

\begin{abstract}
  This document is ``open source'', you can find the \LaTeX{} sources at
  \url{https://github.com/NaoPross/DigDes}. All diagrams were made with TikZ.
  The content is based on the material of Prof. Dr. Zbinden, from the course
  \emph{Digital Design} at the University of Applied Sciences Eastern
  Switzerland (OST). If you find typos or errors you can open an PR on Github
  or mail me at \href{mailto:\theauthoremail}{\texttt{\theauthoremail}} if I'm still
  around (until spring 2022) or \href{mailto:np@0hm.ch}{\texttt{np@0hm.ch}}.
\end{abstract}

\tableofcontents

\section*{License}
\doclicenseThis{}

% \newpage
\twocolumn

\setcounter{page}{1}
\pagenumbering{arabic}

\section{Development model}
The workflow for the development is show in figure
\ref{fig:gajski-kuhn-ychart}. In the Gajski-Kuhn Y-model has 3 axis for the
perspectives of the product. It is typical to start from the behavioral axis,
by treating the systems as a black-box, and then to jump back and forth between
the other axis while gravitating towards the origin (project goal).
\begin{figure}[h]
  \centering
  \resizebox{\linewidth}{!}{
    \begin{tikzpicture}[
        font=\ttfamily,
      ]
      \draw[gray] node[
        circle,
        fill = gray,
        minimum size = 2mm,
        outer sep = 0,
        inner sep = 0,
      ] (O) at (0,0) {};

      \foreach \r/\desc in {
        1/{Electrical},
        2/{Logic gates},
        3/{Register transfer},
        4/{Architecture},
        5/{System}
      }{
        \draw[gray, dashed] (O) circle (\r);
        \draw[gray] (90:\r)
          node[
            above = 1mm,
            align = center,
            font = \small\ttfamily,
            fill = white,
          ] {\desc};
      }

      \draw[hsr-mauve, thick] (O) to ++(30:6)
        node[
          above = 3mm,
          align = center,
          text width = 2cm
        ] {\textbf{Structural Perspective}};
      \foreach \r/\desc in {
        1/{Transistors, Wires},
        2/{Gates, Latches, Flip-Flops},
        3/{ALUs, Registers, Memories},
        4/{Subblocks},
        5/{Top blocks, I/O}
      } {
        \draw[hsr-mauve] (30:\r)
          node[
            circle, minimum size = 2mm,
            fill = hsr-mauve,
            outer sep = 0,
            inner sep = 0
          ] {}
          node[
            below right = 2mm,
            align = left,
            font = \small\ttfamily,
            fill = white,
          ] {\desc};
      }

      \draw[hsr-blue, thick] (O) to ++(150:6)
        node[
          above = 3mm,
          align = center,
          text width = 2cm
        ] {\textbf{Behavioral Perspective}};
      \foreach \r/\desc in {
        1/{Transfer functions},
        2/{Truth tables, State graphs},
        3/{Data moves and operations},
        4/{Subtasks},
        5/{Algorithm}
      } {
        \draw[hsr-blue] (150:\r)
          node[
            circle, minimum size = 2mm,
            fill = hsr-blue,
            outer sep = 0,
            inner sep = 0
          ] {}
          node[
            below left = 2mm,
            align = right,
            font = \small\ttfamily,
            fill = white,
          ] {\desc};
      }

      \draw[hsr-lakegreen, thick] (O) to ++(270:6)
        node[
          below = 3mm,
          align = center,
          text width = 2cm
        ] {\textbf{Physical Perspective}};
      \foreach \r/\desc in {
        1/{Mask polygons, Detailed layout},
        2/{Standard cells, Macro cells},
        3/{Placement and routing},
        4/{Floorplan partitioning},
        5/{Chip or board}
      } {
        \draw[hsr-lakegreen] (270:\r)
          node[
            circle, minimum size = 2mm,
            fill = hsr-lakegreen,
            outer sep = 0,
            inner sep = 0
          ] {}
          node[
            right = 2mm,
            align = left,
            font = \small\ttfamily,
            fill = white,
          ] {\desc};
      }

    \end{tikzpicture}
  }
  \caption{
    Gajski-Kuhn Y-chart.
    \label{fig:gajski-kuhn-ychart}
  }
\end{figure}

%% TODO: finish picture
\iffalse
Figure \ref{fig:asic-design-flow} shows a typical flow diagram of how an ASIC device
is designed.
\begin{figure}[h]
  \begin{tikzpicture}[
      scale = .7,
      font = \small\ttfamily,
      bubble/.style = {
        rectangle,
        draw = black, thick,
        fill = lightgray!10,
        align = center,
        text width = 2.1cm,
        rounded corners = 5pt,
      },
      box/.style = {
        rectangle,
        draw = black, thick,
        fill = lightgray!10,
        align = center,
        text width = 2.1cm,
      },
      lib/.style = {
        rectangle,
        draw = black, gray, thick, dashed,
        fill = lightgray!10,
        align = center,
        text width = 2.1cm,
      },
      ghost/.style = {
        outer sep = 0,
        inner sep = 0,
      }
    ]
    \matrix[row sep = 5mm, column sep = 5mm]{
      \node[bubble] (dd) {Design Description}; & & \node[bubble] (tbd) {Testbench Description}; \\
      \node[ghost] (A) {}; & \node[box] (fs) {Functional Simulation}; \\
      \node[box] (so) {Synthesis \& Optimization}; & \node[lib] (se) {Standard Elements}; \\
      \node[ghost] (lineA) {}; & & \node[ghost] (lineB) {}; \\
      \node[box] (tm) {Technology Mapping}; \\
      \node[box] (ts) {Test Synthesis}; \\
      \node[bubble] (gates) {}; & \node[box] (pres) {Prelayout Simulation}; \\
      \node[box] (l) {Layout}; & \node[box] (posts) {Postlayout Simulation}; \\
      \node[bubble] (design) {}; & & \node[lib] {Technology Library}; \\
    };
  \end{tikzpicture}
  \caption{
    Design flow for an ASIC device.
    \label{fig:asic-design-flow}
  }
\end{figure}
\fi

% \section{Hardware}

\section[VHSIC Hardware Description Language (VHDL)]{
  VHSIC Hardware Description Language (VHDL)
  % \texttt{VHDL}: Very high-speed integrated circuits program
  % Hardware Description Language
}

\subsection{Basic syntax and identifiers}
In VHDL an identifier is a case insensitive string composed of
\texttt{A-Z a-z 0-9 \_} that
\begin{itemize}
  \item is not a keyword,
  \item does not start with a number or \texttt{\_},
  \item does not have two or more \texttt{\_} in a row.
\end{itemize}
Expressions are terminated by a semicolon \texttt{;}.
Two dashes in a row cause the rest of the line to be interpreted as a comment.
\begin{lstlisting}[language=vhdl]
expression; -- comment
\end{lstlisting}

\subsection{Entities and Architectures}
In VHDL the concept of \emph{entity} describes a black box of which only
inputs and outputs are known. The internals of an entity are described through
an \emph{architecture}. There can be multiple architectures for a single entity.

\begin{center}
  \ttfamily
  \begin{tikzpicture}[
      node distance = 1mm,
      pin/.style = {
        draw = black, fill = white, circle, thick,
        inner sep = 0pt, minimum size = 2mm,
      },
    ]
    \node[
      rectangle, draw = black, thick, fill = gray!20!white,
      minimum width = 4.5cm, minimum height = 4cm,
    ] (entity) {};

    \node[anchor = south west] at (entity.north west) {Entity};

    \foreach \x in {1,2,3}{
      \node[
        rectangle, draw = black, thick, fill = white,
        minimum width = 3.5cm, minimum height = .75cm,
      ] (arch\x) at ($(entity.north) + (0, -1 * \x)$) {Architecture \x};
    }

    \foreach \x in {1,...,4}{
      \draw[thick]
        ($(entity.north west) - (0, .75 * \x)$) 
        node[pin] {} to ++(-.75, 0) node (pinl\x) {};

      \draw[thick]
        ($(entity.north east) - (0, .75 * \x)$)
        node[pin] {} to ++(.5, 0) node (pinr\x) {} ;
    }

    \node[right] at (pinr1) {Pin};
  \end{tikzpicture}
\end{center}

Entities are declared with \vhdl{port()} that may contain a list of pins. Pins
have a mode that can be \vhdl{in} input (only LHS\footnote{Left hand side}),
\vhdl{out} output (only RHS\footnote{Right hand side}), \vhdl{inout}
bidirectional or \vhdl{buffer} that can stay both on LHS and RHS. The usage of
the latter is discourareged in favour of an internal signal.
\begin{lstlisting}[language=vhdl]
entity `\reqph{name}` is
  port(
    `\reqph{pin}` : `\reqph{mode} \reqph{type}`;
  );
end `\reqph{name}`;
\end{lstlisting}

Architectures are normally named after the design model, example are
\texttt{behavioral}, \texttt{structural}, \texttt{selective}, etc.
\begin{lstlisting}[language=vhdl]
architecture `\reqph{name}` of `\reqph{entity}` is
  -- declare used variables, signals and component types
begin
  -- concurrent area
end `\optionalph{name}`;
\end{lstlisting}

\subsection{Electric types and Libraries}
VHDL provides some types such as
\begin{itemize}
  \item \vhdl{boolean} true or false,
  \item \vhdl{bit} 0 or 1,
  \item \vhdl{bit_vector} one dimensional array of bits,
  \item \vhdl{integer} 32-bit binary representation of a value.
\end{itemize}
From external libraries other types are available:
\begin{itemize}
  \item \vhdl{std_logic} advanced logic with 9 states,
  \item \vhdl{std_ulogic}
\end{itemize}
The above are from the \vhdl{ieee.std_logic_1164} library, and can take the
values described in the following table.
\begin{center}
  \begin{tabularx}{\linewidth}{>{\ttfamily}c l X}
    \toprule
    Value & Meaning & Usage \\
    \midrule
    U & Uninitialized  & In the simulator \\
    X & Undefined      & Simulator sees a bus conflict \\
    0 & Force to 0     & Low state of outputs \\
    1 & Force to 1     & High state of outputs \\
    Z & High Impedance & Three state ports \\
    W & Weak Unknown   & Simulator sees weak a bus conflict \\
    L & Weak Low       & Open source outputs with pull-down resistor \\
    H & Weak High      & Open drain output with pull-up resistor \\
    - & Don't care     & Allow minimization \\
    \bottomrule
  \end{tabularx}
\end{center}
%% TODO: copy conflict resolutiontable

\subsection{Declarations} \label{sec:declarations}
Before a \vhdl{begin} -- \vhdl{end} block, there is usually a list of declarations.
A self evident examples are \emph{constants}.
\begin{lstlisting}[language=vhdl]
constant `\reqph{name}` : `\reqph{type}` := `\reqph{value}`;
\end{lstlisting}

Next, \emph{signals} and \emph{variables}. Signals is are wires, they can only be
connected and do not have an initial state. Variables can be assigned like in
software, but can cause the synthesization of an unwanted D-Latch.

\begin{lstlisting}[language=vhdl]
signal `\reqph{name}`, `\optionalph{name, \ldots}` : `\reqph{type}`;

variable `\reqph{name}`, `\optionalph{name}`, `\optionalph{\ldots}` : `\reqph{type}`;
variable `\reqph{name}` : `\reqph{type}` := `\reqph{expression}`;
\end{lstlisting}

For the hierarchical designs, when external entities are used, they must be
declared as components. The \vhdl{port()} expression must match the entity
declaration.
\begin{lstlisting}[language=vhdl]
component `\reqph{entity name}` is
  port(
    `\optionalph{list of pins}`
  );
end component;
\end{lstlisting}
For entities with multiple architectures, it is possible to choose which
architecture is used with the following expression.
\begin{lstlisting}[language=vhdl]
for `\reqph{label or {\tt all}}`: use entity `\reqph{library}`.`\reqph{entity}`(`\reqph{architecture}`);
\end{lstlisting}

\subsection{Concurrent Area}
\begin{center}
  \ttfamily
  \begin{tikzpicture}[
      node distance = 1mm,
      pin/.style = {
        draw = black, fill = white, circle, thick,
        inner sep = 0pt, minimum size = 2mm,
      },
      component/.style = {
        draw = black, thick, fill = white, rectangle,
        minimum width = 18mm, minimum height = 12mm,
        align = center,
      },
    ]

    \node[
      draw = black, rectangle, fill = gray!20!white, thick,
      minimum width = .75\linewidth, minimum height = 4cm,
    ] (arch) {};

    \node[anchor = south west] at (arch.north west) {Architecture};

    \node[pin] (clk) at ($(arch.north west) - (0,1)$) {};
    \node[pin] (a) at ($(clk) - (0,1)$) {};
    \node[pin] (b) at ($(a) - (0,1)$) {};
    \node[pin] (y) at ($(arch.north east) - (0,1)$) {};

    \node[left = of clk] {clk};
    \node[left = of a] {a};
    \node[left = of b] {b};
    \node[right = of y] {y};

    \node[component] (c1) at ($(clk) + (2,-.2)$) {Process};
    \node[component] (c2) at ($(c1) + (.2,-1.8)$) {Component\\ Entity};
    \node[
      component, minimum width = 0mm, minimum height = 0mm,
    ] (c3) at ($(c1) + (2.4,-.2)$) {Logic\\ Gate};

    \draw[thick]
      (clk)     to[out = 0, in = 180] ($(c1.west) + (0,.2)$)
      (a)       to[out = 0, in = 180] ($(c1.west) - (0,.2)$)
      (b)       to[out = 0, in = 180] (c2.west)

      (c1.east) to[out = 0, in = 180] ($(c3.west) + (0,.2)$)
      (c2.east) to[out = 0, in = 180] ($(c3.west) - (0,.2)$)
      (c3.east) to[out = 0, in = 180] (y)
      ;

  \end{tikzpicture}
\end{center}

In the architecture between \vhdl{begin} and \vhdl{end}, the expressions
are \emph{not} read sequentially, everything happens at the same time.
Statements inside the concurrent area optionally have a label.
\begin{lstlisting}[language=vhdl]
`\optionalph{label}`: `\reqph{concurrent statement}`;
\end{lstlisting}
In the concurrent area signals, components and processes can be used to create
a logic.

\subsubsection{Signal assignment and simple gates}
Signals are assigned using \vhdl{<=}.
\begin{lstlisting}[language=vhdl]
`\optionalph{label}`: `\reqph{signal}` <= `\reqph{expression}`;
\end{lstlisting}
Simple logic functions such as \vhdl{not}, \vhdl{and}, \vhdl{or}, \vhdl{xor},
etc. can be used.
\begin{lstlisting}[language=vhdl]
 y <= (a and s) or (b and not(s));
\end{lstlisting}

\subsubsection{Aggregates}
For vector types it is possible to create a value out of multiple signals.
\begin{lstlisting}[language=vhdl]
`\reqph{vector}` <= (
  `\reqph{index}`  => `\reqph{source or value}`,
  `\reqph{index}`  => `\reqph{source or value}`,
  `\optionalph{\tt others}` => `\reqph{source or value}`
);
\end{lstlisting}
\begin{lstlisting}[language=vhdl]
-- declaration
signal data : bit_vector(6 downto 0);
signal a, b : bit;
-- concurrent
data = (1 => a, 0 => b, others => '0')
\end{lstlisting}

\subsubsection{Selective and conditional assignment}
Higher level conditions can be written in two ways. 
\begin{lstlisting}[language=vhdl]
-- using when
`\optionalph{label}:` y <= `\reqph{source}` when `\reqph{condition}` else
     `\reqph{source}` when `\reqph{condition}` else
     `\reqph{source}` when `\reqph{condition}`;
\end{lstlisting}
\begin{lstlisting}[language=vhdl]
-- using with
`\optionalph{label}`: with `\reqph{signal}` select `\reqph{dest}` <= 
  `\reqph{source}` when `\reqph{value}`,
  `\reqph{source}` when `\reqph{value}`,
  `\reqph{source}` when others;
\end{lstlisting}

\subsubsection{Components}
External components that have been previously declared can be used with the
\vhdl{port map(}\reqph{assignments}\texttt{)} syntax. For example:
\begin{lstlisting}[language=vhdl]
-- declaration
component flipflop is
  port(
    clk, set, reset : in  std_ulogic,
    Q, Qn           : out std_ulogic 
  );
end component flipflop;

signal clk_int, a, b : in  std_ulogic;
signal y, z          : out std_ulogic;
\end{lstlisting}
\begin{lstlisting}[language=vhdl]
-- concurrent
u1: flipflop
  port map(
    clk   => clk_int,
    set   => a,
    reset => b,
    Q     => y,
    Qn    => z
  );

\end{lstlisting}

\subsubsection{Processes}
For more sophisticated logic VHDL offers a way of writing sequential statements
called \emph{process}.
\begin{lstlisting}[language=vhdl]
`\optionalph{label}:` process (`\optionalph{sensitivity list}`)
-- declarations
begin
  -- sequential statements
end process;
\end{lstlisting}
Processes have a \emph{sensitivity list} that can be empty.  When a signal in
the sensitivity list changes state, the process is executed.  With an empty
sensitivity list, the process runs continuously.  In the declaration,
everything from \S\ref{sec:declarations} applies. For the sequential
statements, the following applies:
\begin{itemize}
  \item Neither selective (\vhdl{with}) nor conditional (\vhdl{when}) should be used.
    They are replaced with new sequential constructs (\vhdl{if} and \vhdl{case}).
  \item Signal assignments (with \vhdl{<=}) change their value
    \emph{only at the end of the process}.
  \item Variables on the other hand change as soon as they are assigned (with \vhdl{:=}).
\end{itemize}
And for good practice:
\begin{itemize}
  \item Before any \vhdl{if} or \vhdl{case} default values should be assigned.
  \item Any signal on the RHS should be in the sensitivity list.
  \item Processes with empty sensitivity lists should only be used for simulations.
\end{itemize}

The sequential replacements for \vhdl{with} and \vhdl{when} are in the listings below.
\begin{lstlisting}[language=vhdl]
if `\reqph{condition}` then
  -- sequential statements 
elsif `\reqph{condition}` then
  -- sequential statements 
else
  -- sequential statements 
end if;
\end{lstlisting}
\begin{lstlisting}[language=vhdl]
case `\reqph{expression}` is
  when `\reqph{choice}` =>
    -- sequential statements
  when `\reqph{choice}` =>
    -- sequential statements
  when others =>
    -- sequential statements
end case;
\end{lstlisting}

Processes can detect \emph{events} of signals. Typically it is used for clocks.
\begin{lstlisting}[language=vhdl]
process (clk)
begin
  -- rising edge
  if clk'event and clk = '1' then
    ... end if;
  if rising_edge(clk) then
    ... end if;

  -- falling edge
  if clk'event and clk = '0' then
    ... end if;
  if falling_edge(clk) then
    ... end if;
end process;
\end{lstlisting}

\subsection{Custom and arithmetic types}
It is possible to create custom types, usually to create state machines.
\begin{lstlisting}[language=vhdl]
type `\reqph{name}` is (`\reqph{identifier}`, `\reqph{identifier}`, `\ph{\ldots}`);
\end{lstlisting}

% vim:ts=2 sw=2 et:

\section{State Machines}
\begin{figure}[h]
  \centering
  \ttfamily
  \begin{tikzpicture}[
      node distance = 3mm,
      box/.style = {
        draw = black, thick, fill = gray!20!white,
        minimum width = 20mm, minimum height = 8mm,
      }
    ]

    % mealey
    \begin{scope}
      \node[box] (G) {\large G};
      \node[box, above = of G] (F) {\large F};
      \node[box, below = of G] (Z) {\large Z};

      \node[above = of F] {\large Mealey};

      \draw[very thick, ->, hsr-blue] (F.east) -- ++(1,0) node[right] {oup};
      \draw[very thick, ->, hsr-lakegreen] (G.east) -- ++(.5,0) |- (Z.east);
      \draw[very thick, ->] (Z.west) -- ++(-.5,0) |- ($(G.west) - (0,.25)$);
      \draw[very thick, ->] (G.west) ++ (-.5,-.25) |- ($(F.west) + (0,.25)$);

      \draw[very thick, ->, hsr-mauve] ($(G.west) + (-1.5,.25)$)
        node[left] {inp} -- ++(1.5,0);
      \draw[very thick, ->, hsr-mauve] ($(G.west) + (-1,.25)$) |- ($(F.west) - (0,.25)$);
    \end{scope}

    % moore
    \begin{scope}[yshift = -42mm]
      \node[box] (G) {\large G};
      \node[box, above = of G] (F) {\large F};
      \node[box, below = of G] (Z) {\large Z};

      \node[above = of F] {\large Moore};

      \draw[very thick, ->, hsr-blue] (F.east) -- ++(1,0) node[right] {oup};
      \draw[very thick, ->, hsr-lakegreen] (G.east) -- ++(.5,0) |- (Z.east);
      \draw[very thick, ->] (Z.west) -- ++(-.5,0) |- ($(G.west) - (0,.25)$);
      \draw[very thick, ->] (G.west) ++ (-.5,-.25) |- (F.west);

      \draw[very thick, ->, hsr-mauve] ($(G.west) + (-1.5,.25)$)
        node[left] {inp} -- ++(1.5,0);
    \end{scope}

    % Medwedjew
    \begin{scope}[yshift = -80mm]
      \node[box] (G) {\large G};
      \node[box, below = of G] (Z) {\large Z};

      \node[above = of G, yshift = 3mm] {\large Medwedjew};

      \draw[very thick, ->, hsr-lakegreen] (G.east) -- ++(.5,0) |- (Z.east);
      \draw[very thick, ->, hsr-blue] (Z.west) -- ++(-.5,0) |- ($(G.west) - (0,.25)$);
      \draw[very thick, ->, hsr-blue] (G.west) ++ (-.5,-.25) |- ($(G.east) + (1,.75)$)
        node[right] {oup};

      \draw[very thick, ->, hsr-mauve] ($(G.west) + (-1.5,.25)$)
        node[left] {inp} -- ++(1.5,0);
    \end{scope}
  \end{tikzpicture}
\end{figure}

\subsection{Encoding the state}
For Mealey and Moore machines it is typical to write:
\begin{lstlisting}[language=vhdl]
type state_type is (st_rst, st_a, st_b, st_c, ...);
signal present_state, next_state : state_type;
\end{lstlisting}
The encoding of the state is left to the synthesizer or can be configured in
the graphical interface of the tool.  If a custom encoding is required
(Medwedjew), adding the following generates a custom encoding.
\begin{lstlisting}[language=vhdl]
attribute enum_encoding : string;
attribute enum_encoding of state_type:
  type is "0001 0010 0100 ...";
\end{lstlisting}
Or an equivalent approach is to use a vector subtype and constants.
\begin{lstlisting}[language=vhdl]
subtype state_type is bit_vector(3 downto 0);

constant st_rst : state_type := "0001";
constant st_a   : state_type := "0010";
constant st_b   : state_type := "0100";
...

signal present_state, next_state : state_type;
\end{lstlisting}

\subsection{Updating the state register (\texttt{Z})}
\begin{lstlisting}[language=vhdl]
register_logic: process (clk, rst)
begin
  -- asynchronous reset
  if rst = '1' then
    present_state <= st_rst;

  -- clock
  elsif rising_edge(clk) then
    present_state <= next_state;
  end if;
end process;
\end{lstlisting}

\subsection{Updating the state (\texttt{G})}
\begin{lstlisting}[language=vhdl]
next_state_logic:
process (present_state, `\optionalph{inputs}`)
begin
  -- default value
  next_state <= state_rst;

  case present_state is
    when st_rst =>
      -- reset state logic
      next_state <= `\reqph{state}`;

    when st_a =>
      -- logic using inputs
      next_state <= `\reqph{state}`;

    ...
    when others => null;
  end case;
end process;
\end{lstlisting}

\subsection{Updating the output (\texttt{F})}
Mealey
\begin{lstlisting}[language=vhdl]
output_logic:
process (present_state, `\reqph{inputs}`)
begin
  -- logic with state and inputs
  `\reqph{output}` <= `\reqph{expression}`;
end process;
\end{lstlisting}
Moore
\begin{lstlisting}[language=vhdl]
output_logic: process (present_state)
begin
  case present_state is
    when st_rst =>
      `\reqph{output}` <= `\reqph{value}`;

    ...
  end case;
end process;
\end{lstlisting}
Medwedjew
\begin{lstlisting}[language=vhdl]
output_logic: `\reqph{output}` <= present_state;
\end{lstlisting}

\section{Testing}
To simulate a digial circuit it is possible to write test benches using VHDL.
The code in this section may no longer be synthetisable, and is usually
written by a \emph{test designer}.

\subsection{Simulator}
VHDL simulates digital systems using \emph{delta cycles}.

%% TODO: notes on how delta cycles work, read
%% https://stackoverflow.com/questions/43652630/delta-cycles-and-waveforms

\subsection{Transport delay}
To model a time delay of a signal there are two ways:
\begin{lstlisting}[language=vhdl]
y <= transport `\reqph{expression}` after `\reqph{time}`;
y <= inertial `\reqph{expression}` after `\reqph{time}`;
\end{lstlisting}
When \vhdl{transport} is used, the output changes only exactly after the
specified time, the simulator simply waits. With \vhdl{inertial}, the output is
also delayed, but only if the input lasts more than the specified time.  This
means that for example with a time of \vhdl{10 ns} a pulse of \vhdl{5 ns} will
be ignored. This is much more typical and realistic, thus when unspecified,
\vhdl{after} is interpreted as \vhdl{inertial ... after}.
\begin{lstlisting}[language=vhdl]
y <= `\reqph{expression}` after `\reqph{time}`;
\end{lstlisting}
%% TODO: tikz timing diagram

\subsection{Generate stimuli}
Simple stimuli (signals) are generated using processes. For example a clock
signal done in three ways:
\begin{lstlisting}[language=vhdl]
-- declaration
constant f : integer := 1000;
constant T : time    := 1 sec/f;
signal clk0, clk1, clk2 : std_ulogic;
\end{lstlisting}
\begin{lstlisting}[language=vhdl]
-- concurrent
clock0: process
begin
  clk <= '1'; wait for (T/2);
  clk <= '0'; wait for (T/2);
end process;

clock1: process
begin
  clk1 <= '1';
  loop
    wait for (T/2);
    clk1 <= not clk1;
  end loop;
end process;

-- lazy way
clock2: clk2 <= not clk2 after (T/2);
\end{lstlisting}
One time stimuli can be modelled using the following expression. Note that the
time is absolute.
\begin{lstlisting}[language=vhdl]
tb_sig <= '0',
  '1' after 20 ns,
  '0' after 30 ns, -- 10 ns later
  `\reqph{value}` after `\reqph{time}`;
\end{lstlisting}
Repeating sequences can be created using processes.
\begin{lstlisting}[language=vhdl]
sequence: process
begin
  tb_sig <= '0';
  wait for 20 ns;
  tb_sig <= '1';
  wait for 10 ns;
  ...
end process;
\end{lstlisting}
For loops are also available, and can be synthesised if they run over a finite
range.
\begin{lstlisting}[language=vhdl]
`\optionalph{label}:` for `\reqph{parameter}` in `\reqph{range}` loop
  -- sequential statements
end loop `\optionalph{label}`;
\end{lstlisting}
A concrete example:
\begin{lstlisting}[language=vhdl]
-- declaration
constant n : integer := 3;
signal a, b : std_ulogic_vector(n-1 downto 0);
\end{lstlisting}
\begin{lstlisting}[language=vhdl]
-- sequential
for i in 0 to 2**n -1 loop
  a <= std_ulogic_vector(
            to_unsigned(i, n));
  for k in 0 to 2**n -1 loop
    b <= std_ulogic_vector(
              to_unsigned(k, n));
  end loop;
end loop;
\end{lstlisting}

\subsection{Assertions}
Assertions are used write tests to check that a signal is in the correct state.
\begin{lstlisting}[language=vhdl]
`\optionalph{label}`: assert `\reqph{condition}` report `\reqph{string}` severity `\reqph{severity}`;
\end{lstlisting}
The \vhdl{report} and \vhdl{severity} are optional but strongly advised. The
severity can take one of 4 values: \vhdl{note}, \vhdl{warning}, \vhdl{error},
\vhdl{failure}. Simulations can be configured to stop in when an error of the
desired severity occurrs. An example:
\begin{lstlisting}[language=vhdl]
assert (tb_y = '0') report "error at vector 11" severity error;
\end{lstlisting}

\subsection{A simple but complete Test Bench}


\section{Samples / Templates}

Below is a template for a simple VHDL file.

\begin{lstlisting}[language=vhdl]
library ieee;
use ieee.std_logic_1164.all;
-- declare entities (`\S\ref{sec:vhdl:entities-arch}`)
entity `\reqph{name}` is
  port(`\optionalph{pins}`);
end entity `\reqph{name}`;
-- declare architectures (`\S\ref{sec:vhdl:entities-arch}`)
architecture `\reqph{name}` of `\reqph{entity name}` is
  -- internal signals (`\S\ref{sec:vhdl:declarations}`)
  -- other components (`\S\ref{sec:vhdl:components}`)
  -- declare custom types (`\S\ref{sec:fsm:encode}`)
  -- variables of custom type (`\S\ref{sec:fsm:encode}`)
begin
  -- assignments and processes (`\S\ref{sec:vhdl:concurrent}`)
end architecture `\reqph{name}`;
\end{lstlisting}
And for a test bench
\begin{lstlisting}[language=vhdl]
library ieee;
use ieee.std_logic_1164.all;

-- declare entities (`\S\ref{sec:vhdl:entities-arch}`)
entity `\reqph{name}`_tb is
  -- nothing here
end entity `\reqph{name}`_tb;

architecture tb of `\reqph{name}`_tb is
  -- simulator settings
  constant freq : natural := `\reqph{frequency}`;
  constant time : time := 1 sec / freq;

  -- component of DUT
  component `\reqph{name}` is
    port(
      clk : in std_ulogic;
      `\optionalph{other I/O}`
    );
  end component `\reqph{name}`;

  signal clk_tb : std_ulogic;
  -- more signals for inputs and outputs
begin

  dut: component `\reqph{name}`
    port map(
      clk => clk_tb;
      `\reqph{other I/O}`);

    clk_generator: process
      -- generate clock (`\S\ref{sec:stimuli}`)
      clk_tb <= '1'; wait for (T/2);
      clk_tb <= '0'; wait for (T/2);
    end process;

    stimuli: process
    begin
      -- generate stimuli (`\S\ref{sec:stimuli}`)
      -- for loops, after, etc.
    end;

    response: process
    -- constants for expected outputs
    begin
      wait for 0.9 * T;
      -- assertions (`\S\ref{sec:assertions}`)
      wait for T;
    end process;

end architecture tb;

\end{lstlisting}
% vim:ts=2 sw=2 et:


\end{document}
% vim:ts=2 sw=2 et:
