\section[VHSIC Hardware Description Language (VHDL)]{
  VHSIC Hardware Description Language (VHDL)
  % \texttt{VHDL}: Very high-speed integrated circuits program
  % Hardware Description Language
}

\newcommand{\ph}[1]{\textrm{\textit{#1}}}
\newcommand{\reqph}[1]{\textrm{\textlangle\,\ph{#1}\,\textrangle}}
\newcommand{\optionalph}[1]{\textrm{[\,\ph{#1}\,]}}

\newcommand{\vhdl}[1]{\lstinline[language=vhdl]{#1}}

\subsection{Basic syntax and identifiers}
In VHDL an identifier is a case insensitive string composed of
\texttt{A-Z a-z 0-9 \_} that
\begin{itemize}
  \item is not a keyword,
  \item does not start with a number or \texttt{\_},
  \item does not have two or more \texttt{\_} in a row.
\end{itemize}
Expressions are terminated by a semicolon \texttt{;}.
Two dashes in a row cause the rest of the line to be interpreted as a comment.
\begin{lstlisting}[language=vhdl]
expression; -- comment
\end{lstlisting}

\subsection{Entities and Architectures}
In VHDL the concept of \emph{entity} describes a black box of which only
inputs and outputs are known. The internals of an entity are described through
an \emph{architecture}. There can be multiple architectures for a single entity.

\begin{center}
  \ttfamily
  \begin{tikzpicture}[
      node distance = 1mm,
      pin/.style = {
        draw = black, fill = white, circle, thick,
        inner sep = 0pt, minimum size = 2mm,
      },
    ]
    \node[
      rectangle, draw = black, thick, fill = gray!20!white,
      minimum width = 4.5cm, minimum height = 4cm,
    ] (entity) {};

    \node[anchor = south west] at (entity.north west) {Entity};

    \foreach \x in {1,2,3}{
      \node[
        rectangle, draw = black, thick, fill = white,
        minimum width = 3.5cm, minimum height = .75cm,
      ] (arch\x) at ($(entity.north) + (0, -1 * \x)$) {Architecture \x};
    }

    \foreach \x in {1,...,4}{
      \draw[thick]
        ($(entity.north west) - (0, .75 * \x)$) 
        node[pin] {} to ++(-.75, 0) node (pinl\x) {};

      \draw[thick]
        ($(entity.north east) - (0, .75 * \x)$)
        node[pin] {} to ++(.5, 0) node (pinr\x) {} ;
    }

    \node[right] at (pinr1) {Pin};
  \end{tikzpicture}
\end{center}

Entities are declared with \vhdl{port()} that may contain a list of pins. Pins
have a mode that can be \vhdl{in} input (only LHS), \vhdl{out} output (only
RHS), \vhdl{inout} bidirectional or \vhdl{buffer} that can stay both on LHS and
RHS. The usage of the latter is discourareged in favour of an internal signal.
\begin{lstlisting}[language=vhdl]
entity `\reqph{name}` is
  port(
    `\reqph{pin}` : `\reqph{mode} \reqph{type}`;
  );
end `\reqph{name}`;
\end{lstlisting}

Architectures are normally named after the design model, example are
\texttt{behavioral}, \texttt{structural}, \texttt{selective}, etc.
\begin{lstlisting}[language=vhdl]
architecture `\reqph{name}` of `\reqph{entity}` is
  -- declare used variables, signals and component types
begin
  -- concurrent area
end `\optionalph{name}`;
\end{lstlisting}

\subsection{Electric types and Libraries}
VHDL provides some types such as
\begin{itemize}
  \item \vhdl{boolean} true or false,
  \item \vhdl{bit} 0 or 1,
  \item \vhdl{bit_vector} one dimensional array of bits,
  \item \vhdl{integer} 32-bit binary representation of a value.
\end{itemize}
From external libraries other types are available:
\begin{itemize}
  \item \vhdl{std_logic} advanced logic with 9 states,
  \item \vhdl{std_ulogic}
\end{itemize}
The above are from the \vhdl{ieee.std_logic_1164} library, and can take the
values described in the following table.
\begin{center}
  \begin{tabularx}{\linewidth}{>{\ttfamily}c l X}
    \toprule
    Value & Meaning & Usage \\
    \midrule
    U & Uninitialized  & In the simulator \\
    X & Undefined      & Simulator sees a bus conflict \\
    0 & Force to 0     & Low state of outputs \\
    1 & Force to 1     & High state of outputs \\
    Z & High Impedance & Three state ports \\
    W & Weak Unknown   & Simulator sees weak a bus conflict \\
    L & Weak Low       & Open source outputs with pull-down resistor \\
    H & Weak High      & Open drain output with pull-up resistor \\
    - & Don't care     & Allow minimization \\
    \bottomrule
  \end{tabularx}
\end{center}
%% TODO: copy conflict resolutiontable

\subsection{Declarations} \label{sec:declarations}
Before a \vhdl{begin} -- \vhdl{end} block, there is usually a list of declarations.
A self evident examples are \emph{constants}.
\begin{lstlisting}[language=vhdl]
constant `\reqph{name}` : `\reqph{type}` := `\reqph{value}`;
\end{lstlisting}

Next, \emph{signals} and \emph{variables}. Signals is are wires, they can only be
connected and do not have an initial state. Variables can be assigned like in
software, but can cause the synthesization of an unwanted D-Latch.

\begin{lstlisting}[language=vhdl]
signal `\reqph{name}`, `\optionalph{name, \ldots}` : `\reqph{type}`;

variable `\reqph{name}`, `\optionalph{name}`, `\optionalph{\ldots}` : `\reqph{type}`;
variable `\reqph{name}` : `\reqph{type}` := `\reqph{expression}`;
\end{lstlisting}

For the hierarchical designs, when external entities are used, they must be
declared as components. The \vhdl{port()} expression must match the entity
declaration.
\begin{lstlisting}[language=vhdl]
component `\reqph{entity name}` is
  port(
    `\optionalph{list of pins}`
  );
end component;
\end{lstlisting}
For entities with multiple architectures, it is possible to choose which
architecture is used with the following expression.
\begin{lstlisting}[language=vhdl]
for `\reqph{label or {\tt all}}`: use entity `\reqph{library}`.`\reqph{entity}`(`\reqph{architecture}`);
\end{lstlisting}

\subsection{Concurrent Area}
\begin{center}
  \ttfamily
  \begin{tikzpicture}[
      node distance = 1mm,
      pin/.style = {
        draw = black, fill = white, circle, thick,
        inner sep = 0pt, minimum size = 2mm,
      },
      component/.style = {
        draw = black, thick, fill = white, rectangle,
        minimum width = 18mm, minimum height = 12mm,
        align = center,
      },
    ]

    \node[
      draw = black, rectangle, fill = gray!20!white, thick,
      minimum width = .75\linewidth, minimum height = 4cm,
    ] (arch) {};

    \node[anchor = south west] at (arch.north west) {Architecture};

    \node[pin] (clk) at ($(arch.north west) - (0,1)$) {};
    \node[pin] (a) at ($(clk) - (0,1)$) {};
    \node[pin] (b) at ($(a) - (0,1)$) {};
    \node[pin] (y) at ($(arch.north east) - (0,1)$) {};

    \node[left = of clk] {clk};
    \node[left = of a] {a};
    \node[left = of b] {b};
    \node[right = of y] {y};

    \node[component] (c1) at ($(clk) + (2,-.2)$) {Process};
    \node[component] (c2) at ($(c1) + (.2,-1.8)$) {Component\\ Entity};
    \node[
      component, minimum width = 0mm, minimum height = 0mm,
    ] (c3) at ($(c1) + (2.4,-.2)$) {Logic\\ Gate};

    \draw[thick]
      (clk)     to[out = 0, in = 180] ($(c1.west) + (0,.2)$)
      (a)       to[out = 0, in = 180] ($(c1.west) - (0,.2)$)
      (b)       to[out = 0, in = 180] (c2.west)

      (c1.east) to[out = 0, in = 180] ($(c3.west) + (0,.2)$)
      (c2.east) to[out = 0, in = 180] ($(c3.west) - (0,.2)$)
      (c3.east) to[out = 0, in = 180] (y)
      ;

  \end{tikzpicture}
\end{center}

In the architecture between \vhdl{begin} and \vhdl{end}, the expressions
are \emph{not} read sequentially, everything happens at the same time.
Statements inside the concurrent area optionally have a label.
\begin{lstlisting}[language=vhdl]
`\optionalph{label}`: `\reqph{concurrent statement}`;
\end{lstlisting}
In the concurrent area signals, components and processes can be used to create
a logic.

\subsubsection{Signal assignment and simple gates}
Signals are assigned using \vhdl{<=}.
\begin{lstlisting}[language=vhdl]
`\optionalph{label}`: `\reqph{signal}` <= `\reqph{expression}`;
\end{lstlisting}
Simple logic functions such as \vhdl{not}, \vhdl{and}, \vhdl{or}, \vhdl{xor},
etc. can be used.
\begin{lstlisting}[language=vhdl]
 y <= (a and s) or (b and not(s));
\end{lstlisting}

\subsubsection{Aggregates}
For vector types it is possible to create a value out of multiple signals.
\begin{lstlisting}[language=vhdl]
`\reqph{vector}` <= (
  `\reqph{index}`  => `\reqph{source or value}`,
  `\reqph{index}`  => `\reqph{source or value}`,
  `\optionalph{\tt others}` => `\reqph{source or value}`
);
\end{lstlisting}
\begin{lstlisting}[language=vhdl]
-- declaration
signal data : bit_vector(6 downto 0);
signal a, b : bit;
-- concurrent
data = (1 => a, 0 => b, others => '0')
\end{lstlisting}

\subsubsection{Seiective and conditional assignment}
Higher level conditions can be written in two ways. 
\begin{lstlisting}[language=vhdl]
-- using when
`\optionalph{label}:` y <= `\reqph{source}` when `\reqph{condition}` else
     `\reqph{source}` when `\reqph{condition}` else
     `\reqph{source}` when `\reqph{condition}`;

-- using with
`\optionalph{label}`: with `\reqph{signal}` select `\reqph{dest}` <= 
  `\reqph{source}` when `\reqph{value}`,
  `\reqph{source}` when `\reqph{value}`,
  `\reqph{source}` when others;
\end{lstlisting}

\subsubsection{Components}
External components that have been previously declared, can be used with the
\vhdl{port map(}\reqph{assignments}\texttt{)} syntax. For example:
\begin{lstlisting}[language=vhdl]
-- declaration
component flipflop is
  port(
    clk, set, reset : in  std_ulogic,
    Q, Qn           : out std_ulogic 
  );
end component flipflop;

signal clk_int, a, b : in  std_ulogic;
signal y, z          : out std_ulogic;

-- concurrent
u1: flipflop
  port map(
    clk   => clk_int,
    set   => a,
    reset => b,
    Q     => y,
    Qn    => z
  );

\end{lstlisting}

\subsubsection{Processes}
For more sophisticated logic, VHDL offers a way of writing sequential
statements called \emph{processes}.
\begin{lstlisting}[language=vhdl]
`\optionalph{label}:` process (`\optionalph{sensitivity list}`)
-- declarations
begin
  -- sequential statements
end process;
\end{lstlisting}
Processes have a \emph{sensitivity list} that can be empty.  When a signal in
the sensitivity list changes state, the process is executed.  With an empty
sensitivity list, the process runs continuously.  In the declaration,
everything from \S\ref{sec:declarations} applies. For the sequential
statements, the following applies:
\begin{itemize}
  \item Neither selective (\vhdl{with}) nor conditional (\vhdl{when}) should be used.
    They are replaced with new sequential constructs (\vhdl{if} and \vhdl{case}).
  \item Signal assignments (with \vhdl{<=}) change their value
    \emph{only at the end of the process}.
  \item Variables on the other hand change as soon as they are assigned (with \vhdl{:=}).
\end{itemize}
And for good practice:
\begin{itemize}
  \item Before any \vhdl{if} or \vhdl{case} default values should be assigned.
  \item Any signal on the RHS should be in the sensitivity list.
  \item Processes with empty sensitivity lists should only be used for simulations.
\end{itemize}

The sequential replacements for \vhdl{with} and \vhdl{when} are in the listings below.
\begin{lstlisting}[language=vhdl]
if `\reqph{condition}` then
  -- sequential statements 
elsif `\reqph{condition}` then
  -- sequential statements 
else
  -- sequential statements 
end if;
\end{lstlisting}
\begin{lstlisting}[language=vhdl]
case `\reqph{expression}` is
  when `\reqph{choice}` =>
    -- sequential statements
  when `\reqph{choice}` =>
    -- sequential statements
  when others =>
    -- sequential statements
end case;
\end{lstlisting}

Processes can detect \emph{events} of signals. Typically it is used for clocks.
\begin{lstlisting}[language=vhdl]
process (clk)
begin
  -- rising edge
  if clk'event and clk = '1' then
    ... end if;
  if rising_edge(clk) then
    ... end if;

  -- falling edge
  if clk'event and clk = '0' then
    ... end if;
  if falling_edge(clk) then
    ... end if;
end process;
\end{lstlisting}

\subsection{Custom and arithmetic types}
It is possible to create custom types, usually to create state machines.
\begin{lstlisting}[language=vhdl]
type `\reqph{name}` is (`\reqph{identifier}`, `\reqph{identifier}`, `\ph{\ldots}`);
\end{lstlisting}

% vim:ts=2 sw=2 et:
